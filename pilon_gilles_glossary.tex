\newdualentry{anome}{ANOME}{analysis of main effects}{A chart that compares the average of each level to the overall average}
\newdualentry{anomr}{ANOMR}{analysis of mean ranges}{A chart that compares the average range of each level to the overall range}
\newdualentry{cp}{$C_\text{p}$}{process capability index two-sided (rational sample)}{Compares the width of the process specification to the width of the process variation}
\newdualentry{cpk}{$C_\text{pk}$}{process capability index one-sided (rational sample)}{Calculates the distance from each specification limit to the average and compares then to the width of the process variation}
\newdualentry{cpm}{$C_\text{pm}$}{process capability index loss function (random sample)}{Calculates process capability with respect to deviation from the average and deviation from the target}
\newdualentry{icc}{ICC}{intraclass correlation coefficient}{Describes how strongly units in the same group resemble each other}
\newdualentry{msa}{MSA}{Measurement System Analysis}{Methods to evaluate measurements systems, to understand the causes of measurement variation}
\newdualentry{pp}{$P_\text{p}$}{process capability index two-sided (random sample)}{Compares the width of the process specification to the width of the process variation}
\newdualentry{ppk}{$P_\text{pk}$}{process capability index one-sided (random sample)}{Calculates the distance from each specification limit to the average and compares then to the width of the process variation}
\newglossaryentry{average_chart}
{
    name={average chart},
    description={It is a process behaviour chart that plots the average of a subgroup of values taken in a rational manner. The chart also has an upper control limit line, a lower control limit line, and an average line.},
    plural={average charts}
}
\newglossaryentry{chi_square_distribution}
{
    name={chi-square distribution},
    description={It is the distribution of a sum of the squares of k independent standard normal random variables with k degrees of freedom.},
    plural={chi-square distributions}
}
\newglossaryentry{confidence_interval}
{
    name={confidence interval},
    description={In frequentist statistics, it is a range of estimates for an unknown parameter, interval calculated from values in a sample.
    },
    plural={confidence intervals}
}
\newglossaryentry{dependent_variable}
{
    name={dependent variable},
    description={The result of varying the values of other variables. In $y = mX + b$, y is the dependent variable.},
    plural={dependent variables}
}
\newglossaryentry{distance_to_nearest_specification}
{
    name={distance to nearest specification (sigma level)},
    description={The difference between the location statistic and the closest specification limit, divided by the dispersion statistic (e.g. the standard deviation)}
}
\newglossaryentry{distance_to_target}
{
    name={distance to target},
    description={The difference between the specification target and the average is the difference you must adjust to aim the process correctly.}
}
\newglossaryentry{dot_plot}
{
    name={dot plot},
    description={A chart in which dots are used to depict the quantitative values (e.g. counts) associated with categorical variables},
    plural={dot plots}
}
\newglossaryentry{hypothesis_test}
{
    name={hypothesis test},
    description={A method for proving that an observed difference was not due to random chance},
    plural={hypothesis tests}
}
\newglossaryentry{independent_variable}
{
    name={independent variable},
    description={The values that are deliberately or randomly changed. In $y = mX + b$, x is the independent variable.},
    plural={independent variables}
}
\newglossaryentry{individuals_chart}
{
    name={individuals chart},
    description={It is a process behaviour chart that plots the single values taken in a rational manner. The chart also has an upper control limit line, a lower control limit line, and an average line.},
    plural={individuals charts}
}
\newglossaryentry{moving_range_chart}
{
    name={moving range chart},
    description={It is a process behaviour chart that plots the difference between successive pairs of values taken in a rational manner. The chart also has an upper control limit line, a lower control limit line, and an average line.}
    plural={moving range charts}
}
\newglossaryentry{normal_distribution}
{
    name={normal distribution},
    description={A type of continuous probability distribution for a real-valued random variable},
    plural={normal distributions}
}
\newglossaryentry{outlier}
{
    name={outlier},
    description={It is a data point that differs significantly from other observations. An outlier may be due to variability in the measurement or it may indicate experimental error. An outlier can cause serious problems in statistical analyses.},
    plural={outliers}
}
\newglossaryentry{parallelism_chart}
{
    name={parallelism chart},
    description={It is a process behaviour chart that plots the lines from the average chart and puts them on top of each other. It is used in measurement system analysis. Differences for any part indicate an operator bias relative to other operators.},
    plural={parallelism charts}
}
\newglossaryentry{process_capability}
{
    name={process capability},
    description={The ability of a process to meet a performance standard}
}
\newglossaryentry{process_capability_index}
{
    name={process capability index},
    description={The ratio of a specification measure and a dispersion measure},
    plural={process capability indices}
}
\newglossaryentry{probable_error}
{
    name={probable error},
    description={Also called resolution. It is the interval within which a single measurement might occur 50 \% of the time, $\pm 0.6745 \times \sigma_\text{pure error}$}
}
\newglossaryentry{process_behaviour_chart}
{
    name={process behaviour chart},
    description={It is used to determine if the variation of a sample statistic of a process is stable and predictable, that is, in a state of statistical control. It is also called a Shewhart chart and a control chart.}
    plural={process behaviour charts}
}
\newglossaryentry{range_chart}
{
    name={range chart},
    description={It is a process behaviour chart that plots the range of a subgroup of values taken in a rational manner. The chart also has an upper control limit line, a lower control limit line, and an average line.},
    plural={range charts}
}
\newglossaryentry{sources_of_variation}
{
    name={sources of variation},
    description={These are the factors in a process that cause variation on the dependent variable. These factors may vary in a deliberate manner (fixed effects), random manner (random effects), or mixed effects (fixed and random factors)}
}
\newglossaryentry{specification}
{
    name={specification},
    description={In the context of measurement system analysis, process capability, and process control, a specification is a lower limit, a target, and an upper limit},
    plural={specifications}
}
\newglossaryentry{statistical_control}
{
    name={statistical control},
    description={A process is said to be in a state of statistical control if the variation for both charts is stable and predictable. This means that no points fail various control chart rules, such as Nelson's rules, and no systematic patterns of variation are present. It is also called process stability.}
}
\newglossaryentry{taguchi_loss_function}
{
    name={Taguchi Loss Function},
    description={It is a graphical depiction of loss to describe a phenomenon affecting the value of products produced by a company. It was developed by the Japanese business statistician Genichi Taguchi.}
}
\newglossaryentry{test_retest_error}
{
    name={test-retest error},
    description={The standard deviation of making repeated measurements of the same characteristic on the same parts with the same measuring instrument}
}
\newglossaryentry{type_i_error}
{
    name={type I error},
    description={It is the probability of saying something is different when in fact it is not. We find a difference that isn’t really there. It is also called alpha risk, producer's risk, and false positive.},
    plural={Type I errors}
}
\newglossaryentry{type_ii_error}
{
    name={type II error},
    description={It is the probability of saying something is not different when in fact it is. We fail to find a difference when there is one. It is also called beta risk, consumer’s risk, and false negative. It is used to calculate the power of the test (1-beta), as well as in calculations of sample size.},
    plural={Type II errors}
}
\newglossaryentry{variance}
{
    name={variance},
    description={It is a measure of dispersion, how far a set of numbers is spread out from their average. It is the expectation of the squared devitaion of a random variable from its population average or sample saverage.}
    plural={variances}
}
