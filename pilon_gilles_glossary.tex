\newdualentry{anome}{ANOME}{analysis of main effects}{A chart that compares the average of each level to the overall average}
\newdualentry{anomr}{ANOMR}{analysis of mean ranges}{A chart that compares the average range of each level to the overall range}
\newdualentry{cp}{$C_\text{p}$}{process capability index two-sided (rational sample)}{Compares the width of the process specification to the width of the process variation}
\newdualentry{cpk}{$C_\text{pk}$}{process capability index one-sided (rational sample)}{Calculates the distance from each specification limit to the average and compares then to the width of the process variation}
\newdualentry{cpm}{$C_\text{pm}$}{process capability index loss function (random sample)}{Calculates process capability with respect to deviation from the average and deviation from the target}
\newdualentry{icc}{ICC}{intraclass correlation coefficient}{Describes how strongly units in the same group resemble each other}
\newdualentry{msa}{MSA}{Measurement System Analysis}{Methods to evaluate measurements systems, to understand the causes of measurement variation}
\newdualentry{pp}{$P_\text{p}$}{process capability index two-sided (random sample)}{Compares the width of the process specification to the width of the process variation}
\newdualentry{ppk}{$P_\text{pk}$}{process capability index one-sided (random sample)}{Calculates the distance from each specification limit to the average and compares then to the width of the process variation}
\newglossaryentry{confidence_interval}
{
    name={confidence interval},
    description={It is an interval calculated from values in a sample. The interval is constructed for a specified probability of containing the unknown true value of a population parameter.
    },
    plural={confidence intervals}
}
\newglossaryentry{dependent_variable}
{
    name={dependent variable},
    description={The result of varying the values of other variables. In $y = mX + b$, y is the dependent variable.},
    plural={dependent variables}
}
\newglossaryentry{distance_to_nearest_specification}
{
    name={distance to nearest specification (sigma level)},
    description={The difference between the location statistic and the closest specification limit, divided by the dispersion statistic (e.g. the standard deviation)}
}
\newglossaryentry{dot_plot}
{
    name={dot plot},
    description={A chart in which dots are used to depict the quantitative values (e.g. counts) associated with categorical variables},
    plural={dot plots}
}
\newglossaryentry{hypothesis_test}
{
    name={hypothesis test},
    description={A method for proving that an observed difference was not due to random chance},
    plural={hypothesis tests}
}
\newglossaryentry{independent_variable}
{
    name={independent variable},
    description={The values that are deliberately or randomly changed. In $y = mX + b$, x is the independent variable.},
    plural={dependent variables}
}
\newglossaryentry{normal_distribution}
{
    name={normal distribution},
    description={A type of continuous probability distribution for a real-valued random variable},
    plural={normal distributions}
}
\newglossaryentry{process_capability}
{
    name={process capability},
    description={The ability of a process to meet a performance standard}
}
\newglossaryentry{process_capability_index}
{
    name={process capability index},
    description={The ratio of a specification measure and a dispersion measure},
    plural={process capability indices}
}
\newglossaryentry{probable_error}
{
    name={probable error},
    description={TBD}
}
\newglossaryentry{statistical_control}
{
    name={statistical control},
    description={A process is said to be in a state of statistical control if the variation for both charts is stable and predictable. This means that no points fail various control chart rules and no systematic patterns of variation are present.}
}
\newglossaryentry{test_retest_error}
{
    name={test-retest error},
    description={The standard deviation of making repeated measurements of the same characteristic on the same parts with the same measuring instrument}
}
