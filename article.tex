% This is a template for documentclass{article}.
% See https://ctan.org/ for available packages and documentation.

% must always be first line of a <path.tex> document
\documentclass[10pt, letterpaper, twoside]{article}
\usepackage[utf8]{inputenc}
\usepackage[sfdefault]{noto}
\setlength{\textwidth}{6in}
\setlength{\textheight}{9in}
\setlength{\topmargin}{-1in}
\setlength{\oddsidemargin}{.25in}
\setlength{\evensidemargin}{.25in}

% this package has better control of title, author elements
\usepackage{titling}
\setlength{\droptitle}{-8pt}
\pretitle{\begin{flushleft}\LARGE\bfseries}
\posttitle{\par\end{flushleft}}
\preauthor{\begin{flushleft}\LARGE}
\postauthor{\end{flushleft}}
\predate{\begin{flushleft}}
\postdate{\end{flushleft}}
\date{\today}
\title{article title}
\author{article author}
\setlength{\parindent}{0pt}
\setlength{\parskip}{6pt}
\raggedbottom
\renewenvironment{abstract}{\par\noindent\textbf{\abstractname}\par}

% this package has better control over the date format
\usepackage{datetime2}

% this package has better control over 'floats'
\usepackage{float}

% this package has better control over images
\usepackage{graphicx}
\graphicspath{{fot/}}
\usepackage[export]{adjustbox}
\usepackage[singlelinecheck=false,justification=justified]{caption}

% very useful for mathematics
\usepackage{amsmath, amsthm, amssymb}

% very useful for hyperlinks
\usepackage{hyperref}

% to create a list of equations
\usepackage{tocloft}
\setlength{\cfttabindent}{0in}
\setlength{\cftfigindent}{0in}

% used to create tables
\usepackage{booktabs}

% set references indent

% useful for definitions, theorems, corollaries, etc.
\theoremstyle{myenv}
\theoremstyle{mytheorem}
\newtheorem{theorem}{Theorem}
\newtheorem{corollary}[theorem]{Corollary}
\newtheorem{proposition}[theorem]{Proposition}
\newtheorem{lemma}[theorem]{Lemma}
\newtheorem{axiom}[theorem]{Axiom}
\theoremstyle{mydefinition}
\newtheorem{definition}[theorem]{Definition}
\newtheorem{exercise}[theorem]{Exercise}
\newtheorem{question}[theorem]{Question}
\newtheorem{problem}[theorem]{Problem}
\newtheorem{example}[theorem]{Example}
\newtheorem{remark}[theorem]{Remark}
\makeatletter
\@addtoreset{theorem}{section}
\makeatother

% set no indent for lists
\usepackage{enumitem}
\setlist[itemize]{leftmargin=*}
\setlist[enumerate]{leftmargin=*}

\begin{document}
\maketitle

\begin{abstract}
Lorem ipsum dolor sit amet, consectetur adipiscing elit, sed do eiusmod tempor incididunt ut labore et dolore magna aliqua. Ut enim ad minim veniam, quis nostrud exercitation ullamco laboris nisi ut aliquip ex ea commodo consequat. Duis aute irure dolor in reprehenderit in voluptate velit esse cillum dolore eu fugiat nulla pariatur. Excepteur sint occaecat cupidatat non proident, sunt in culpa qui officia deserunt mollit anim id est laborum.
\end{abstract}

% create a list of equations command
\newcommand{\listequationsname}{List of Equations}
\newlistof{myequations}{equ}{\listequationsname}
\newcommand{\myequations}[1]{%
\addcontentsline{equ}{myequations}{\protect\numberline{\theequation}#1}\par}
\renewcommand{\cftequtitlefont}{\normalfont\Large\bfseries}

% create a list of theorems command
\newcommand{\listtheoremsname}{List of Theorems}
\newlistof{mytheorems}{thm}{\listtheoremsname}
\newcommand{\mytheorems}[1]{%
\addcontentsline{thm}{mytheorems}{\protect\numberline{\thetheorem}#1}\par}
\renewcommand{\cftthmtitlefont}{\normalfont\Large\bfseries}

% create a list of definitions command
\newcommand{\listdefinitionsname}{List of Definitions}
\newlistof{mydefinitions}{def}{\listdefinitionsname}
\newcommand{\mydefinitions}[1]{%
\addcontentsline{def}{mydefinitions}{\protect\numberline{\thedefinition}#1}\par}
\renewcommand{\cftdeftitlefont}{\normalfont\Large\bfseries}

% create a list of corollaries command
\newcommand{\listcorollariesname}{List of Corollaries}
\newlistof{mycorollaries}{cor}{\listcorollariesname}
\newcommand{\mycorollaries}[1]{%
\addcontentsline{cor}{mycorollaries}{\protect\numberline{\thecorollary}#1}\par}
\renewcommand{\cftcortitlefont}{\normalfont\Large\bfseries}

\tableofcontents

\newpage
\section{Introduction}
\label{sec:introduction}
Lorem ipsum dolor sit amet, consectetur adipiscing elit, sed do eiusmod tempor incididunt ut labore et dolore magna aliqua. Ut enim ad minim veniam, quis nostrud exercitation ullamco laboris nisi ut aliquip ex ea commodo consequat. Duis aute irure dolor in reprehenderit in voluptate velit esse cillum dolore eu fugiat nulla pariatur. Excepteur sint occaecat cupidatat non proident, sunt in culpa qui officia deserunt mollit anim id est laborum.

\section{Prior research}
\label{sec:prior_research}
Lorem ipsum dolor sit amet, consectetur adipiscing elit, sed do eiusmod tempor incididunt ut labore et dolore magna aliqua. Ut enim ad minim veniam, quis nostrud exercitation ullamco laboris nisi ut aliquip ex ea commodo consequat. Duis aute irure dolor in reprehenderit in voluptate velit esse cillum dolore eu fugiat nulla pariatur. Excepteur sint occaecat cupidatat non proident, sunt in culpa qui officia deserunt mollit anim id est laborum.

\section{Results}
\label{sec:results}
Lorem ipsum dolor sit amet, consectetur adipiscing elit, sed do eiusmod tempor incididunt ut labore et dolore magna aliqua. Ut enim ad minim veniam, quis nostrud exercitation ullamco laboris nisi ut aliquip ex ea commodo consequat. Duis aute irure dolor in reprehenderit in voluptate velit esse cillum dolore eu fugiat nulla pariatur. Excepteur sint occaecat cupidatat non proident, sunt in culpa qui officia deserunt mollit anim id est laborum.

\newpage
\section{Bold, italics, and underline}
\label{sec:bold_italics_underline}
Some of the \textbf{greatest} \textit{discoveries} in \underline{science} were made by \textbf{\textit{accident}}.

\section{URLs}
\label{sec:urls}
Hyperlinks within the document can be easily added. For example, see Graphics \ref{sec:graphics} and Anatomy of a matplotlib figure \ref{fig:anatomy_of_a_matplotlib_figure} and Definition of a limit \ref{def:limit}.

Hyperlinks to external URLs can be easily added. For example, see \href{https://en.wikipedia.org/wiki/Latex}{Wikipedia \LaTeX}.
\section{Lists}
\subsection{Unordered list}
\label{sec:unordered_list}
\begin{itemize}
    \item Illinois
        \begin{itemize}
            \item Chicago
            \item Naperville
        \end{itemize}
    \item Kentucky
        \begin{itemize}
            \item Lexington
            \item Louisville
        \end{itemize}
\end{itemize}

\subsection{Ordered list}
\label{sec:ordered_list}
\begin{enumerate}
    \item Cat
        \begin{enumerate}
            \item Persian
            \item Siamese
        \end{enumerate}
    \item Dog
        \begin{enumerate}
            \item Aussiedoodle
            \item Newfoundland
        \end{enumerate}
\end{enumerate}

\newpage
\section{Graphics}
\label{sec:graphics}
Lorem ipsum dolor sit amet, consectetur adipiscing elit, sed do eiusmod tempor incididunt ut labore et dolore magna aliqua. Ut enim ad minim veniam, quis nostrud exercitation ullamco laboris nisi ut aliquip ex ea commodo consequat. Duis aute irure dolor in reprehenderit in voluptate velit esse cillum dolore eu fugiat nulla pariatur. Excepteur sint occaecat cupidatat non proident, sunt in culpa qui officia deserunt mollit anim id est laborum.

% H means 'here definitely' to place the figure
\begin{figure}[H]
% \includegraphics[width=\textwidth, inner]{matplotlib_anatomy_of_a_figure}
\includegraphics[width=3in, left]{matplotlib_anatomy_of_a_figure}
\caption{Anatomy of a matplotlib figure}
\end{figure}
\label{fig:anatomy_of_a_matplotlib_figure}

Lorem ipsum dolor sit amet, consectetur adipiscing elit, sed do eiusmod tempor incididunt ut labore et dolore magna aliqua. Ut enim ad minim veniam, quis nostrud exercitation ullamco laboris nisi ut aliquip ex ea commodo consequat. Duis aute irure dolor in reprehenderit in voluptate velit esse cillum dolore eu fugiat nulla pariatur. Excepteur sint occaecat cupidatat non proident, sunt in culpa qui officia deserunt mollit anim id est laborum.

\newpage
\section{Tables}
\label{sec:tables}

\begin{table}[H]
\begin{flushleft}
\begin{tabular}{lcccccc}
\toprule
& I &  II & III & IV & V & VI \\
\midrule
Vandali     & 123 & 456 & 678 & 321 & 644 & 768  \\
Visigothorum & 021 & 229 & 678 & 123 & 456 & 678 \\
\bottomrule
\end{tabular}
\label{tab:visigothi_cum_romanis}
\caption{Visigothi cum Romanis}
\end{flushleft}
\end{table}

\begin{table}[H]
\begin{flushleft}
\begin{tabular}{ccc}
\toprule
$A$ & $B$ & $ A \land B$ \\
\midrule
T & T & T \\
T & F & F \\
F & T & F \\
F & F & F \\
\bottomrule
\end{tabular}
\label{tab:boolean_truth_table}
\caption{Boolean truth table}
\end{flushleft}
\end{table}

\newpage
\section{Mathematics}
\label{sec:mathematics}

\subsection{Inline mathematics}
\label{sec:inline_mathematics}`
Examples of inline mathematics $\frac{a}{b}$, followed by text and another inline equation $\lim_{x\to 0}\frac{\sin x}{x}$=1 and more text $\sum_{k=0}^\infty \frac{1}{2^k}=2$ and finally much more text to cover many lines of a paragraph.

\subsection{Display-style mathematics}
\label{sec:display_style_mathematics}
Examples of mathematics on their own lines. All equations are flush-left to the page unless multilines, and thus aligned on a symbol with the longest equation flush-left. The equations numbers are flush-right.

\begin{flalign}
    \text{measured observation} = \text{true value} + \text{measurement system error}&&
\end{flalign}
\myequations{Measured observation}

\begin{flalign}
    \text{ICC} = \frac{\text{variance of the product values}}{\text{variance of the product observations}}&&
\end{flalign}
\myequations{Intraclass correlation coefficient}

\begin{flalign}
    \overline{X}  \pm \text{CI} & = \frac{\sum\limits_{i=1}^{n} X_i}{n} \pm t_{1- \alpha / 2 \, \text{,} \, n-1} \times \frac{s}{\sqrt{n}}&&
\end{flalign}
\myequations{Average $\pm$ confidence interval}

\begin{flalign}
    \text{distance to target} & = \text{specification target} - \text{location statistic} = T - \overline{X}&&
\end{flalign}
\myequations{Distance to target}

\begin{flalign}
    s \pm \text{CI} & = \sqrt{\frac{\sum\limits_{i=1}^{n} \left(X_i - \overline{X}\right)^2}{n - 1}} \pm \left(\sqrt{\frac{(n - 1)\,s^{2}}{\chi_{\alpha / 2}^2}},\sqrt{\frac{(n - 1)\,s^{2}}{\chi_{1 - {\alpha / 2}}^2}}\right)&&
\end{flalign}
\myequations{Standard deviation $\pm$ confidence interval}

\begin{flalign}
    \text{distance to nearest specification} & = \frac{\text{nearest specification} - \text{location statistic}}{\text{dispersion statistic}}&& \\
                                             & = \frac{\text{min}\,(\overline{X} - \text{LSL}\,, \text{USL} - \overline{X})}{s}&&
\end{flalign}
\myequations{Distance to nearest specification}

The confidence intervals are:

\begin{flalign}
    \text{Lower bound} = \hat{P}_{pk} - z_{1 - \alpha / 2} \sqrt{\frac{1}{\left(\frac{\text{toler}}{2}\right)^2 n} + \frac{\left(\hat{P}_{pk}\right)^2}{2 \nu}}&&
\end{flalign}
\myequations{$\hat{P}_{pk}$ lower bound}

\begin{flalign}
    \text{Upper bound} = \hat{P}_{pk} + z_{1- \alpha / 2} \sqrt{\frac{1}{\left(\frac{\text{toler}}{2}\right)^2 n} + \frac{\left(\hat{P}_{pk}\right)^2}{2 \nu}}&&
\end{flalign}
\myequations{$\hat{P}_{pk}$ upper bound}

where:

\begin{flalign*}
    n                           & = \text{the number of observations in the sample}&& \\
    \nu                         & = \text{the degrees of freedom $= n - 1$}&& \\
    \chi_{\alpha \text{,}\nu}^2 & = \text{the $\alpha$ percentile of the chi-square distribution with $\nu$ degrees of freedom}&& \\
    \text{toler}                & = \text{multiplier of the standard deviation tolerance (Minitab\textregistered\ uses 6 as the default value)}&& \\
    z_{1 - \alpha / 2}          & = \text{the $1 - \alpha / 2$ percentile of the standard normal distribution}&& \\
\end{flalign*}

\section{Definitions}
\label{sec:definitions}

\begin{definition}[Definition of limit]
\label{def:limit}
Let $f$ be a function defined on an open interval containing $x=a$, but perhaps not at $x=a$.  We say $\displaystyle\lim_{x\to a}f(x)=L$ if for any $\epsilon>0$ there exists $\delta>0$ such that $|f(x)-L|<\epsilon$ whenever $0<|x-a|<\delta$.
\end{definition}
\mydefinitions{Definition of limit}

\begin{definition}
\label{def:derivative}
The \textit{derivative} of $f$ is defined to be $\displaystyle f^\prime(x)=\lim_{h\to 0}\frac{f(x+h)-f(x)}{h}$, if this limit exists.
\end{definition}
\mydefinitions{Derivative of f}

\section{Theorems}
\label{sec:theorems}

\begin{theorem}[Fundamental Theorem of Calculus]
\label{thm:ftc}
If $f$ is continuous on the interval $[a,b]$, then \[\frac{d}{dx}\int_a^x f(t)dx=f(x)\] for all $a\le x\le b$.
\end{theorem}
\mytheorems{Fundamental theorem of calculus}

\section{Corollaries}
\label{sec:corollaries}
\begin{corollary}
If $f$ is continuous on the interval $[a,b]$, then \[\int_a^b f(x)dx=F(b)-F(a)\] where $F$ is any antiderivative of $f$.
\end{corollary}
\mycorollaries{f is continuous}

\newpage
\section{Citations}
\label{sec:citations}
The formatting of the bibliography is not quite what I wish. I will explore bibtex for this. For now, here is how to site articles, books, and online references.

Blah blah \cite{JD} and blah blah \cite{JS} and blah blah \cite{JT}.

\newpage
\listoffigures

\newpage
\listoftables

\newpage
\listofmyequations

\newpage
\listofmydefinitions

\newpage
\listofmytheorems

\newpage
\listofmycorollaries

\newpage
\section{Acknowledgements}
\label{sec:acknowledgements}

The author wishes to thank ...

\section{Supporting information}
\label{sec:supporting_information}

Add files, data, and other supporting information, and where it can be obtained.

\begin{thebibliography}{99}
\bibitem{JD} Doe, John. 2000. Title of book.
\bibitem{JS} Smith, John. 2013. \textit{Title of article}.
\bibitem{JT} Thomas, John. 2022. \textit{Online reference}.
\end{thebibliography}

\end{document}

